%%% This is a Markdown GitHub styled design template %%%

\documentclass[9pt,a4paper]{report}

\usepackage[french]{babel}  % french language package
\frenchbsetup{StandardLists=true}  % english bullet lists
\usepackage{fontspec}
\usepackage[margin=1cm]{geometry}  % 1 centimetres margin

% Main font of the document:
\setmainfont{SF-Pro-Text}[Path = fonts/,
    					  UprightFont = *-Regular,
    					  BoldFont = *-Semibold,
					  	  ItalicFont = *-RegularItalic,
						  BoldItalicFont = *-SemiboldItalic]

% Mono font of the document:
\setmonofont{SFMono}[Path = fonts/,
					 UprightFont = *-Regular,
					 BoldFont = *-Bold,
					 ItalicFont = *-RegularItalic,
					 BoldItalicFont = *-BoldItalic,
					 Scale=0.8]

% Setting heading font for sections and subsections:
\newfontfamily\headingfont{SF-Pro-Display}[Path = fonts/,
										   UprightFont = *-Regular,
										   BoldFont = *-Semibold,
										   ItalicFont = *-RegularItalic,
										   BoldItalicFont = *-SemiboldItalic]

%% In case fontspec doesn't work with accents:
%\usepackage{polyglossia}
%\setmainlanguage{french}

\usepackage{url, hyperref}
\usepackage{verbatim}

\usepackage[table]{xcolor}

\usepackage[framemethod=TikZ]{mdframed}

\usepackage{tabularx}
\usepackage{array, multirow}
\usepackage{colortbl}  % table colours
\usepackage{tcolorbox}

\usepackage{listings}  % package listings
\usepackage{enumitem}

\usepackage[normalem]{ulem}  % scratched text

\usepackage{titlesec}

\newlength\tindent
\setlength{\tindent}{\parindent}
\setlength{\parindent}{0pt}
\renewcommand{\indent}{\hspace*{2.5em}}

\definecolor{darkgreen}{rgb}{0,0.6,0}
\definecolor{mauve}{rgb}{0.58,0,0.82}
\definecolor{darkred}{RGB}{167,29,93}
\definecolor{darkblue}{RGB}{24,54,145}

% New colours %
\definecolor{deep-gray}{RGB}{36,41,46}
\definecolor{darkgray}{RGB}{50,50,50}
\definecolor{gray}{RGB}{106,115,125}
\definecolor{table-gray}{RGB}{223,226,229}
\definecolor{light-gray}{RGB}{234,236,239}
\definecolor{code-gray}{RGB}{244,244,245}
\definecolor{block-gray}{RGB}{246,248,250}
\definecolor{red}{RGB}{215,58,73}
\definecolor{blue}{RGB}{0,92,197}
\definecolor{dark-blue}{RGB}{3,47,98}
\definecolor{purple}{RGB}{111,66,193}
\definecolor{orange}{RGB}{227,98,9}


% Old colours
\definecolor{cyan}{RGB}{93,217,239}
\definecolor{orange}{RGB}{253,151,31}
\definecolor{rose}{RGB}{249,38,114}
\definecolor{yellow}{RGB}{230,219,116}
\definecolor{green}{RGB}{82,190,91}
\definecolor{lime}{RGB}{166,226,46}
\definecolor{darkorange}{RGB}{233,83,37}

\color{deep-gray}  % general colour for the whole document

% Inline code frames:
\newcommand{\codex}[1]{\colorbox{code-gray}{\texttt{#1}}}  % inline code with sharp corners
\newcommand\code[2][fill=code-gray]{\tikz[baseline]\node[inner ysep=1pt, inner xsep=2.5pt, anchor=text, rectangle, rounded corners=2pt, #1]{\strut\texttt{#2}};}  % inline code with rounded corners
\newtcbox{\term}{on line, fontupper=\ttfamily\color{block-gray}, colback=deep-gray, boxrule=0pt, arc=2pt, boxsep=0pt, left=3pt, right=3pt, top=3pt, bottom=2.5pt} % inline terminal styled code

% Commands for syntax highlighting:
\newcommand{\key}[1]{\textcolor{red}{\texttt{#1}}}  % keyword
\newcommand{\func}[1]{\textcolor{purple}{\texttt{#1}}}  % function declaration
\newcommand{\param}[1]{\textcolor{deep-gray}{\texttt{\textsl{#1}}}}  % parameter
\newcommand{\built}[1]{\textcolor{blue}{\texttt{#1}}}  % built-in function
\newcommand{\op}[1]{\textcolor{red}{\texttt{#1}}}  % operator
\newcommand{\ent}[1]{\textcolor{blue}{\texttt{#1}}}  % number
\newcommand{\str}[1]{\textcolor{dark-blue}{\texttt{#1}}}  % string
\newcommand{\argm}[1]{\textcolor{orange}{\texttt{#1}}}  % keyword argument
\newcommand{\com}[1]{\textcolor{gray}{\texttt{#1}}}  % line comment

\mdfsetup{font=\ttfamily,
		  backgroundcolor=block-gray,
		  roundcorner=3pt,
		  hidealllines=true,
		  leftmargin=0,
		  rightmargin=0,
		  innerleftmargin=10,
		  innerrightmargin=30,
		  innertopmargin=10,
		  innerbottommargin=10}
		  

%----------------------------------------------------------------------------------------
%	CODE INCLUSION CONFIGURATION
%----------------------------------------------------------------------------------------

% -- Python configuration --
% Set the delimiter for classes declarations
\newcommand{\classClassHighlight}[1]{\textcolor{red}{class} \textcolor{purple}{#1}:}
% Set the delimiter for functions declarations
\newcommand{\functionDefHighlight}[1]{\textcolor{red}{def} \textcolor{purple}{#1}(}
% Set the delimiter for special methods declarations
\newcommand{\specialMethodHighlight}[1]{\textcolor{red}{def} \textcolor{blue}{\_\_#1\_\_}}

\lstdefinelanguage{Python} {
	% Standard keywords:
    morekeywords=[1]{and, as, assert, break, class, continue, def, del, elif, else, except,
    				 finally, for, from, global, if, import, in, is, lambda, nonlocal, not,
				 	 or, pass, raise, return, try, while, with, yield},
    %
    % Built-in functions:
    morekeywords=[2]{abs, all, any, ascii, bin, bool, bytearray, bytes, callable, chr,
    				 classmethod, compile, complex, delattr, dict, dir, divmod, enumerate,
				 	 eval, exec, False, filter, float, format, frozenset, getattr, globals,
					 hasattr, hash, help, hex, id, input, int, isinstance, issubclass,
					 iter, len, list, locals, map, max, memoryview, min, next, None,
					 object, oct, open, ord, pow, print, property, range, repr, reversed,
					 round, self, set, setattr, slice, sorted, staticmethod, str, sum,
					 super, True, tuple, type, vars, zip},
    %
    % Operators:
    % (The colour of otherkeywords is the same colour as keywordstyle[1])
    otherkeywords={+, -, *, /, =, <, >, \%, +=, -=, *=},
    %
    morecomment=[l][\color{gray}]{\#}, % Python comments (#)
    %
    morestring=[b]", % Strings defined with "
    morestring=[b]', % Strings defined with '
    %
    moredelim=[is][\classClassHighlight]{class\ }{:}, % Delimiter for classes declarations
	moredelim=[is][\functionDefHighlight]{def\ }{(}, % Delimiter for functions declarations
	moredelim=[is][\specialMethodHighlight]{def\ __}{__}, % Delimiter for special methods declarations
	moredelim=[s][\color{blue}]{__}{__}, % Delimiter for special methods
}


% -- Pseudo-code configuration --
% Set the delimiter for structure declarations
\newcommand{\structureHighlight}[1]{\textcolor{red}{structure} \textcolor{purple}{#1} \{}
% Set the delimiter for procedure (with arguments) declarations
\newcommand{\procedureHighlight}[1]{\textcolor{purple}{#1}(}
% Set the delimiter for functions declarations
\newcommand{\procedureWithoutArgumentsHighlight}[1]{\textcolor{red}{procédure} \textcolor{purple}{#1} \{}

\lstdefinelanguage{Pseudo-code} {
	% Mots-clés standards :
    morekeywords=[1]{alors, et, faire, non, ou, procédure, que, retourner, si, sinon,
    				 structure, tant},
    %
    % Fonctions intégrées :
    morekeywords=[2]{afficher, Faux, longueur, max, min, None, somme, Vrai},
    %
    % Operators:
    % (The colour of otherkeywords is the same colour as keywordstyle[1])
    otherkeywords={+, -, *, /, =, <, >, \%, +=, -=, *=},
    %
	morecomment=[l][\color{gray}]{//}, % Commentaires en ligne (//)
    morecomment=[l][\color{gray}]{\#}, % Commentaires python (#)
    %
    moredelim=[is][\structureHighlight]{structure\ }{\ \{}, % Délimiteur pour les déclarations de structure
	moredelim=[is][\procedureHighlight]{procedure\ }{(}, % Délimiteur pour les déclarations de procédure
	moredelim=[is][\procedureWithoutArgumentsHighlight]{procédure\ }{\ \{}, % Délimiteur pour les déclarations de procédure sans arguments
    morestring=[b]", % Chaînes de caractères définies avec : "
    morestring=[b]', % Chaînes de caractères définies avec : '
    %
}


% -- JavaScript configuration --
\lstdefinelanguage{JavaScript}{
  keywords={typeof, new, true, false, catch, function, return, null, catch, switch, var,
  			if, in, while, do, else, case, break},
  morekeywords=[2]{class, export, boolean, throw, implements, import, this},
  sensitive=false,
  comment=[l]{//},
  morecomment=[s]{/*}{*/},
  morestring=[b]',
  morestring=[b]"
}

                                          
% -- General configuration --
\lstset {
	frame=none, % No frame around code
	backgroundcolor=\color{block-gray}, % Background colour
	basicstyle=\linespread{1.1}\ttfamily, % Line spread and ttfamily font
	upquote=true, % Straight single quotes
	columns=flexible, % Flexible column alignment
	keepspaces=true, % Not drop spaces to fix column alignment and always convert tabulators to spaces
	commentstyle=\color{gray}, % Comments are gray
	stringstyle=\color{dark-blue}, % Strings are dark-blue
	showstringspaces=false, % Don't put marks in string spaces
	tabsize=4, % 4 spaces per tab
	%
	keywordstyle=[1]\color{red}, % Keywords style
	keywordstyle=[2]\color{blue}, % Built-in functions style
	%
	% Colours for numbers:
	% (The star indicates that literate replacements should not be made in strings, comments, and other delimited text)
	literate=*{0}{{{\color{blue}0}}}{1}
			  {1}{{{\color{blue}1}}}{1}
			  {2}{{{\color{blue}2}}}{1}
			  {3}{{{\color{blue}3}}}{1}
			  {4}{{{\color{blue}4}}}{1}
			  {5}{{{\color{blue}5}}}{1}
			  {6}{{{\color{blue}6}}}{1}
			  {7}{{{\color{blue}7}}}{1}
			  {8}{{{\color{blue}8}}}{1}
			  {9}{{{\color{blue}9}}}{1},
	%
	numbers=left, % Line numbers on left
	firstnumber=1, % Line numbers start with line 1
	numberstyle=\tt\scriptsize\color{gray}, % Line numbers are blue and small
	%stepnumber=5, % Line numbers go in steps of 5
}

\surroundwithmdframed[
  hidealllines=true, % Hide mdframed frame line border
  innerleftmargin=25pt, % Margin between mdframed's left border and lstlisting's numbers
  innertopmargin=3pt,
  innerbottommargin=3pt]{lstlisting}
  
% Creates a new command to include a python script:
% The first parameter is the filename of the script (without .py) and the second parameter is the caption
\newcommand{\pythonscript}[2]{
\begin{itemize}
\item[]\lstinputlisting[caption=#2, label=#1]{#1.py}
\end{itemize}
}

\titleformat{\section}{\huge\bfseries\headingfont}{\thesection}{1em}{}[{\color{light-gray}\titlerule[0.5pt]}]  % horizontal rule under sections
\titleformat{\subsection}{\LARGE\bfseries\headingfont}{\thesubsection}{1em}{}[{\color{light-gray}\titlerule[0.5pt]}]  % horizontal rule under subsections
\titleformat{\subsubsection}{\Large\bfseries\headingfont}{\thesubsection}{1em}{}[{\color{light-gray}}]
\titleformat*{\paragraph}{\large\bfseries\headingfont}
%\titleformat*{\subparagraph}{\large\bfseries}

\titlespacing*{\section}{0pt}{0.5cm}{0.5cm}  % 0.5cm above and below the section
\titlespacing*{\subsection}{0pt}{0.5cm}{0.5cm}  % 0.5cm above and below the subsection

\usepackage{setspace}  % package for line spacing
\onehalfspacing  % line spacing of one and a half

\setcounter{secnumdepth}{0} % no section numbering
