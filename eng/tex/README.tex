\input{preamble}
\usepackage{tabu}

\begin{document}


\section{Schoten - the really smart home implementation with Lucy}

Lucy is the smart home software that delivers an integrated experience
for all home control purposes. It is voice controlled by Amazon Alexa,
Ok Google and Apple Siri and drives many interfaces such as Lutron,
Philips Hue lights, Ikea Tradfri and vera control ltd Vera.

My house in Schoten is implemented with this Lucy, and these site
implementation files are stored here. These are private to the owner of
the house and as much serve as an implementation example of Lucy as a
full functional house solution.

There is just 1 configuration file that define a site implementation,
\url{site.conf}.

This readme file covers the structure of this file and has click through
to all the underlying support documents.

\section{site.conf : the Portable Home Configuration File (PHCF)}

\subsection{Config Parser Structure}

This file is a configuration file that contains all the home definitions
and descriptors. It implements a basic configuration language which
provides a structure similar to what’s found in Microsoft Windows INI
files. This file can be customized by end users if they have some
computer knowledge and text file editing skills.

For more information on config parsing, see
https://docs.python.org/3/library/configparser.html

The syntax of site.conf is checked by a config parser program by every
raspberry node in the network and is the basis of the data structures
used by Lucy's scripts and an extensive html report is available once
parsing is completed.

The file app\_obj.conf in the Lucy package (not the home package)
contains all the fields and their purpose for every interface or
control. The config parser reads this definitions to validate the
parsing of site.conf, so it is the best source for information and
validation.

The first 3 sections (square braced words in the beginning of the line)
in site.conf are mandatory and should be in that order, then the rooms
or places are defined (in square braces):

\begin{enumerate}
\def\labelenumi{\arabic{enumi}.}
\item
  \textbf{{[}DEFAULT{]}} : do not change, this is a config parser thing
\item
  \textbf{{[}VERSION{]}} : version of the config file, update this after
  making changes
\item
  \textbf{{[}APPS{]}} : the system apps and the drivers, the functional
  apps can only be defined in the room or place where they provide their
  functionality
\item
  \_\_{[}ROOM/PLACE{]}*\_\_ : the sections where rooms (in-house) or
  places (out-house) definitions are specified
\end{enumerate}

Comments must be prefixed with \# or ;

Example (... means must continue):

\begin{verbatim}
[DEFAULT]       # is applied automatically to every room
    is_room=True
    security={"zone":"house"}
[VERSION]       # should be first, is mandatory and contains the version information of this file
    is_room=False
    version="h-3.0.0"
[APPS]    # should be after version, is mandatory and contains all non room linked configuration, interfaces and drivers
    is_room=False
    site_settings={           
            "latitude":             51.2849951,                 # of your location.  It is used for sunrise / sunset calculations (-90..90)
            "longitude":            4.5322251,                  # -180..180
            "currency":             "€&euro;",                  # 1 to 3 characters the currency letters and the HTML currency text (must start with &)
            "degrees":              "°C",                       # °C or °F
            # **** Below are the special status virtuals, is_holiday does not imply that that alarm system is armed, this should be done seperately
            "is_reboot":            Virtual("app", duration=1),   # do not change the duration
            "is_day":               Virtual("app", value_logic={"00:00":"False","sunrise+00:15":"True","sunset-00:10":"False"}),
            "EMAIL_prime":          "rudy_vandenberghe@hotmail.com",
            "EMAIL_other":          "lucyaraujo13@hotmail.com",
            "language":             "english", # not yet implemented
            "IP_WAN":               "8.8.8.8", # not yet implemented
            "site_id":              "schoten"}
    ...
[STREET]
...
[GARDEN]
...
[KITCHEN]
...
\end{verbatim}

Firstly, the core apps are outlined, followed with the system apps, the
drivers and then the functional apps. Then base things follow which are
the basis of all things which are presented last.

Potentially you can skip the core apps and first visit the rest, and
come back to these if you need more detail.

\subsection{Core Apps}

\begin{tabular}{ l l }
what\_is & Description \\
\hline
\href{config\_parser} (docs/config\_parser.md) & Config Parsing verifies and turns your site.conf file into program data \\
\href{multi\_app} (docs/config\_parser.md) & Multiple Occurrences for complex functional apps such climate\_system or security\_system or weather\_station \\
\href{notifications} (docs/Notifications.md) & Notifications are triggered by an app or a thing and post their message through different channels \\
\href{pin\_definitions} (docs/pin.md) & the reference to the physical aspect of a thing, such as an input, an output or sensors \\
\href{properties} (docs/properties.md) & more detailed information on item properties, adding features and logic \\
\href{rooms\_places} (docs/room\_place.md) & define rooms, subrooms, places and zones and link this all together \\
\href{things} (docs/devices.md) & definable input things or output things \\
\href{things\_sync} (docs/things\_sync.md) & The process of ensuring that the as\_is and to\_be state of a Thing is there where it needs to be for the system to work as a whole \\
\href{value\_logic} (docs/value\_logic.md) & value logic : value setting/blocking based on conditions or time events \\
\href{virtual} (docs/Virtual.md) & Virtuals are Virtual Things and they make your configuration smart and easy to understand \\
\end{tabular}

\subsection{Multi App occurrences}

To allow multiple instances of security and climate system, the options
can be indexed by adding an index trailer: \_x with x a number and add
to the raspberry roles a keyword 'option\_x' with x the index number, so
that only these options will be parsed for the raspberry that will
handle that set of security or climate arrangements.

Example of 2 climate systems managed by 2 separate raspberry's:

\begin{verbatim}
climate_system_0={"role_me":"PI-Climate",..}
climate_system_1={"role_me":"PI-Climate2",..}
raspi={
    "PI-Climate":  Raspi(ip="192.168.15.70",io_dev="UniPi,24,14",color="magenta", roles=["option_0"]),
    "PI-Climate2": Raspi(ip="192.168.15.71",io_dev="UniPi,24,14",color="magenta", roles=["option_1"]),
}
\end{verbatim}

This works also for other interfaces such as irrigation, weather, etc..

\subsection{Properties : adding features and logic}

Properties is a keyword allowed in the definition of Inputs, Outputs,
Lights, Buttons, Dimmers,.. to define a dictionary of additional
attributes. The fields of this dictionary are checked for eligibility
and it can be best showed through some examples.

Simple Example to have a voice message when internet access is lost or
restored:

\begin{verbatim}
"internet_lost":
    Virtual("app", properties={
        "say_active":"tts_internet_lost",
        "say_normal":"tts_internet_there"})
\end{verbatim}

A more complex example involves the use of virtuals for post mail
handling.\\
It will trigger a voice and mail message when new mail arrives and
create an email message when the mail is removed. At the same time a
virtual "have\_mail" is maintained and set when there is new mail and
lowered when the mail is taken.

\begin{verbatim}
mailbox_monitor={  # It is connected to 2 magnetic switches which register incoming and retrieved post mail
    "role_me": "PI-Mail",                                               
    "mail_in": Input(pin="pi:PI-Mail,0", properties={
        "say_active":   "tts_mail",
        "have_mail":    Virtual("do_me_up"),
        "nty_active":   "nty_mail_in"}),
    "mail_out": Input(pin="pi:PI-Mail,1", properties={
        "have_mail":    Virtual("do_me_down"),
        "nty_active":   "nty_mail_out"})}
\end{verbatim}

Nesting of properties is allowed to a few levels deep, but it is best to
keep nesting levels shallow.

see \href{docs/door.md}{a gate entrance door example} for a complex
nesting structure and the extensive use of virtuals and notifications.

Although this may look like programming, you can easily add features
without changing any of the programming scripts.

Please study the site.conf example for tricks and features you may need.

\subsection{Virtuals and Events : make your configuration smart}

When pondering the strategy to separate home specific configuration from
the programming scripts, the idea emerged of using virtuals and events
to make the configuration smart and to have a powerful design mechanism
to have logic in the site.conf configuration file. It worked out better
than i anticipated and the result is a very powerful combination. The
negatives, cluttering the configuration file with virtuals has proven to
be manageable although the syntax of closing parentheses and braces can
drive you almost nuts. However, it clearly delivers the result of having
stable and separated processing code.

as an example of frequently used virtual actions (many more are
available):


\href{docs/Virtual.md}{Further reading on how to use virtuals.}

\href{docs/Event.md}{Further reading on how to use events.}

\subsection{Value\_Logic : set item values based on conditions or time
events}

Home automation results in devices getting values based on very complex
circumstances and value\_logic is a reasonable simple solution to
capture this complexity in a (hopefully) readable manner.

Two situations can be defined, one whereby the device get overruled and
a certain value is set or the other situation whereby the value of the
device is frozen until this lock is lifted.

Value logic is a value setting dictionary that can be defined in output
derived devices, see \href{docs/value_logic.md}{value\_logic}

A few value\_logic examples will clarify:

\begin{enumerate}
\def\labelenumi{\arabic{enumi}.}
\item
  below is\_day is defined as True after sunrise and before sunset else
  False

\begin{verbatim}
"is_day": Virtual("app", value_logic={"00:00":"False","sunrise":"True","sunset":"False"})
\end{verbatim}
\item
  here the curtains open 15 minutes after sunrise and before sunset or
  5pm whatever comes first and freezelist the item the rest of the time.

\begin{verbatim}
"vera_curtains": Virtual("do_me_follow", 
    value_logic={"sunrise+00:15":True,"00:00":False,"07:30":True,"17:00":False,"sunset":False})
\end{verbatim}
\item
  this example sets the ventilation speed based on the occupancy modes
  of the house if it is day or night

\begin{verbatim}
"ventilation_speed": Dimmer(pin="pi:PI-WWaste,0", 
    value_logic={"is_holiday":"0.10","is_armed_full":"0.10","is_armed_partial":"0.20","is_day":"0.50"})
\end{verbatim}
\item
  this example sets the hall light off when nobody is there, if sleeping
  then the value is 10 else 25 before sunset else 100

\begin{verbatim}
"hall_light": Dimmer(pin="pi:AR-Hall,0", 
    value_logic={"is_armed_partial":"10","is_armed_full":"0","is_holiday":"0","00:00":"25","sunset":"100"})
\end{verbatim}
\item
  for philips hue items, a hue defined scene name can be used in the
  value\_logic, see below example where the light will be off when the
  room is protected or before sunrise. Then it will be at
  spring\_blossom at 20 (values are 0..255) till noon, then
  tropical\_twilight at 150 and finally from sunset savanna\_sunset at
  50.

\begin{verbatim}
"office_ledstrip": Dimmer(pin="hue:Hue_Bridge,{:}",
    value_logic={"is_room_secure":"0",
            "00:00":"0",
            "sunrise":"20,spring_blossom",
            "12:00":"150,tropical_twilight",
            "sunset":"50,savanna_sunset"}, 
    properties={"type":"RGB"})
\end{verbatim}
\end{enumerate}

\subsection{Referencing Things}

Firstly what are input or output derived objects and what are allowed
pin types?

In all input or output derived objects, the definition of a
\textbf{pin=} is mandatory.

Pin types define the source of the input / output element and they have
each to follow a specific syntax:



For tcp and udp, 'ip\_txt' string can contain python string formatting,
whereby the variables are \{name\} for substituting the device name and
\{val\} the current value of the device.

Pin definition Examples:

\begin{verbatim}
"iButton_out_red":     Light(pin="pi:AR-1W-OUT,1",duration=2)           # pin is an arduino, output pin 1
"door_kitchen_terras": Door(pin="1w:A,12742ECC000000B4")                # pin is DS2406 pin A 
"HVAC_test":           Clim_SP(pin="zw:p_switch,135,1",i_make={"wind"}) # pin is zw device 135 variable 1
"office_wall_light":   Light(pin="vera:{:}")                            # pin is vera device named office_wall_light
"guest_light":         Dimmer(pin="ikea:Ikea_Tradfri,{:}")              # pin is tradfri device guest_light
"bed_candle":          Dimmer(pin="hue:Hue_Bridge2,{:}")                # pin is on Hue Bridge 2, named bed candle
"temp_outside":        Sensor(pin="289574D906000075")                   # pin is 1w temp sensor 
"tcp_light":           Light(pin="PI-Light/tcp:192.168.15.76,5000,'{name}-{val}'") # pin is tcp message 
"udp_light":           Light(pin="PI-Light/udp:192.168.15.76,5000,'{val}'")        # pin is udp message
\end{verbatim}

\subsection{Notifications}

Using a prefix in the properties of an option or an io item, certain
notifications can be defined.

As example, in \href{docs/irrigation.md}{irrigation} there is an
attribute nty\_all, that has a list of names that are notifications that
will be called automatically when that particular irrigation event
happens.

With a prefix, these notification triggers can be activated in
PI\_Home.conf in the option irrigation to allow the following:


So to make a speech notification for the completion of irrigation, the
identifier "tts\_irr\_completed" should be referenced in PI\_Home.conf
in irrigation, and "tts\_irr\_completed" should be defined in
PI\_Install.conf, section {[}TTS{]} in the tts\_msg dictionary.

Some examples:

\begin{itemize}
\item
  for range devices we have val\_x (x the value), example whereby a text
  message is played through sonos when climatisation switches from
  heating to cooling :
\end{itemize}

\begin{verbatim}
"climate_mode":         
    Flag_R("role",[-1,0,1], ["Cooling","Off","Heating"]), 
        properties={
            "tts_val_-1":   "tts_now_cooling",
            "tts_val_1":    "tts_now_heating"})
\end{verbatim}

\begin{itemize}
\item
  for binary devices we have active and normal. In this example there is
  a zwave multi switch device that get triggered when the flag do\_sonos
  gets active and a tcp and udp message when this flag becomes non
  active.
\end{itemize}

\begin{verbatim}
"do_sonos": 
    Flag("role", value_logic={"is_armed":"False","is_holiday":"False","00:00":"False","09:15":"True","22:30":"False"}, 
        properties={
            "zw_active":    "zw:p_switch,135,1",    # this is the vera virtual switch Voice Anotations kept in sync
            "zw_normal":    "zw:p_switch,135,0"}),
            "tcp_normal":   "tcp:192.168.15.58,5000,'text!!'", 
            "udp_normal":   "udp:192.168.15.58,5000,'text!!'"})
\end{verbatim}

\begin{itemize}
\item
  for sensors we have high/low and normal. In this example there is a
  notification email and a sonos speech message when the temperature is
  high and back to normal.
\end{itemize}

\begin{verbatim}
"Case°C_aslave_rear":   
    Sensor("28FF7C1965040004",
        properties={
            "check_event": Event("check_float", 
                parameters={
                    "low":  None,
                    "high": 40.0,
                    "nty_high":    "nty_box_high",
                    "nty_normal":  "nty_box_normal",
                    "say_high":    "tts_box_high",
                    "say_normal":  "tts_box_normal"})})
\end{verbatim}

see \href{docs/notifier.md}{notifier} for more details on the
notification syntax.

\section{app\_data.conf : the site install configuration file}

This configuration file is home site specific but contains information
that is typically only for the installer of the system.

Sections of this file:


Apache License, Version 2.0

Raspberry Pi is a trademark of the Raspberry Pi Foundation


\end{document}
